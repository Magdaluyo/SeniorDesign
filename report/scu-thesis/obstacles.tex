\chapter{Obstacles Encountered}

The main obstacles we encountered within our design involved the SensorTag device from Texas Instruments. In trying to use the Zigbee firmware, we were unable to communicate between the SensorTag and XBee. Using SensorTag’s SmartRF Packet Sniffer 2 software, we were able to configure the SensorTag as a sniffer module. We found the packets to be very different from one another. In the SensorTag packet (Figure \ref{fig:sensortag}), it is very simple with a length value, sequence number, and data following the start frame delimiter. The XBee packet (Figure \ref{fig:xbee}) looks largely different. The packet is shown in hex as some values are not human-readable. It starts with a sequence number (0x1d), has various hex values, and then data (0x61 - 0x65). Even more peculiar, we find there to be a second sequence number (0x08). The data between the sequence numbers is unknown to us. With a few parameter changes to the XBee we are able to modify the packet a little bit but we were unable to have it coincide with the SensorTag’s packet which was not modifiable.

\begin{figure}[h]
  \includegraphics[width=0.8\textwidth]{SensorTag-packet.png}
  \centering
  \caption{SensorTag 802.15.4 Packet}
  \label{fig:sensortag}
\end{figure}

\begin{figure}[h]
  \includegraphics[width=0.9\textwidth]{XBee-packet.PNG}
  \centering
  \caption{XBee 802.15.4 Packet}
  \label{fig:xbee}
\end{figure}

We decided that it might be possible to use Bluetooth instead. The SensorTag could connect directly to the Raspberry Pi Zero W and reduce a cost of our system by not using the XBee. Unfortunately, we found SensorTag to be a difficult device to program. The IDE had a steep learning curve and documentation was difficult. Changes in documentation were not always reflected elsewhere causing confusion.