\chapter{Requirements}

\section{Description}
In this chapter, we list the functional and nonfunctional requirements prioritizing them from top to bottom. In section 2.2, we list functional requirements by critical, recommended, and suggested and these describe what the system will do. In section 2.3, we list non-functional requirements in the same way and these describe how the system will perform. Lastly, in section 2.4, we discuss our design constraint and how it affects the scope of this project.

\section{Functional Requirements}
\begin{itemize} 
\item Critical
	\begin{itemize} 
	\item Doorbell will activate light system
	\end{itemize}
\item Recommended
	\begin{itemize} 
	\item Doorbell will activate vibration system
	\item Doorbell is integratabtle with other light/vibration devices available
	\item The System will be low power
	\end{itemize}
\item Suggested
	\begin{itemize} 
	\item Doorbell will not communicate over WiFi
	\end{itemize}
\end{itemize}
\section{Non-Functional Requirements}
\begin{itemize} 
\item Critical
	\begin{itemize} 
	\item The system will be cheap, ideally \$50 or less for a full house integration
	\end{itemize}
\item Recommended
	\begin{itemize} 
	\item The system will be able to connect to various types of devices to allow preferred devices per user
	\end{itemize}
\item Suggested
	\begin{itemize} 
	\item The system has an appealing aesthetic
	\end{itemize}
\end{itemize}
\section{Constraints}
The constraint of our design is that we are not building from the ground up. This makes achieving our goal of an affordable system very difficult as we will be using already built parts. As such is the case, we will have to design this system as a proof of concept.