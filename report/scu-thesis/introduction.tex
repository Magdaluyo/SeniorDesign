\chapter{Introduction}

\section{Problem and Motivation}

Doorbells and knocking are often useless for those that are audibly impaired. Someone with hearing impairments needs to depend on other residents to answer the door for them or purchase expensive equipment to send them phone notifications. Depending on the neighborhood, the ability to know there’s a visitor at the door is essential for quality of life including community involvement and package delivery.

The inspiration for this system came from one of our homes. Since the doorbell stopped functioning within certain parts of the home, it was easy to see the impact not knowing someone is at the door can have on a resident due to the inability to hear a doorbell. This also created a burden on others within the house due to repeated doorbell presses and searching for person the visitor intends to speak with. 

\section{Solution}
In this section, we look at current solutions that exist and provide insight into our solution.

\subsection{Current Solutions}
Current solutions are limited in supporting these individuals. Louder doorbells are useless to those that are completely deaf and can be bothersome for housemates that are not hearing impaired. Some solutions like Physen’s Doorbell Kits are viable and affordable solutions but they are not scalable. Physen and other manufacturers make closed systems that offer limited options. There are also doorbells such as the Ring’s doorbell which can send a notification to a phone; however, this solution can be expensive. Ring’s security camera doorbell costs \$100. That can be well above the financial capabilities of many people. Especially among deaf individuals, we find that unemployment is high at 47\% \cite{pdf:deaf-employment} and many in this population are senior citizens; therefore, finding an affordable solution is very important. There are also scalable options such as the IFTTT platform, which can connect various IoT devices that are readily available on their platform. Unfortunately, this system is closed. Companies must be registered to their system and developed for it. The IFTTT platform also drops and adds devices to the system without warning. The devices available to help hearing impaired individuals in the IFTTT platform can be expensive such as connecting the Ring doorbell to Philips Hue lights which can add up to at least \$200.

%Older version of this section
%Current solutions are limited in supporting these individuals. Louder doorbells are useless to those that are completely deaf and can be annoying for housemates that are not hearing impaired. Current light systems are not integrated into the house and must be carried around the house (or multiple installments are needed, thereby increasing the cost). Furthermore, the light systems may be easily missed if not facing the device or if in a well-lit room. Video doorbells are expensive and require a smartphone to interface with it. The notifications may be missed during the time frame where a person is at the door. A useful alternative is smart wearables but those are limited in battery life. For example, the Apple Watch Series 4 is advertised to last 18 hours on average use \cite{website:apple-watch-battery} which is not useful for a person when sleeping. Designing a low-cost solution was important to the team, as unemployment among deaf individuals is high at 47\% \cite{pdf:deaf-employment}. Generally, hard of hearing individuals are senior citizens who do not have much money. A simple and affordable solution would improve their lives greatly. The hearing impaired community already struggle enough and providing an affordable option to doorbells is essential in keeping them aware of their surroundings and being part of their communities.

\subsection{Contribution}
Our solution is a proof of concept for a scalable and affordable system. Our doorbell is low power and connects to a modular system. A gateway consisting of a Raspberry Pi Zero W and an XBee will be able to connect any device a user desires by using Node-RED.
%To tackle these issues, our solution would aim at integrating a system into the entire household by using Philips Hue light fixtures or, as a cheaper alternative, providing LED strips, removing the need for a portable device that may be easily forgotten. Furthermore, the system will give the user options on a configuration that works best for their home. On top of that we will provide a low power vibration bracelet solution. This would be sleek so that it can be worn while sleeping while providing a gentle but noticeable vibration. The bracelet could be charged while away from the house, so a user will always be able to use it once in the house. Together, our system will be low cost while providing more functionality towards the hearing impaired.
