\chapter{Introduction}

\section{Problem and Motivation}

Doorbells and knocking are often useless for those that are audibly impaired. Someone with hearing impairments needs to depend on other residents to answer the door for them or purchase expensive equipment to send them phone notifications. Depending on the neighborhood, the ability to know there’s a visitor at the door is essential for quality of life including package delivery and community involvement.

\section{Solution}
This section will look at the current solutions that exist and describe how this system will improve on current implementations.

\subsection{Current Solutions}
Current solutions are limited in supporting these individuals. Louder doorbells are useless to those that are completely deaf and can be annoying for housemates that are not hearing impaired. Current light systems are not integrated into the house and must be carried around the house (or multiple installments are needed, thereby increasing the cost). Furthermore, the light systems may be easily missed if not facing the device or if in a well-lit room. Video doorbells are expensive and require a smartphone to interface with it. The notifications may be missed during the time frame where a person is at the door. A useful alternative is smart wearables but those are limited in battery life. For example, the Apple Watch Series 4 is advertised to last 18 hours on average use \cite{website:apple-watch-battery} which is not useful for a person when sleeping. Designing a low-cost solution was important to the team, as unemployment among deaf individuals is high at 47\% \cite{pdf:deaf-employment}. Generally, hard of hearing individuals are senior citizens who do not have much money. A simple and affordable solution would improve their lives greatly. The hearing impaired community already struggle enough and providing an affordable option to doorbells is essential in keeping them aware of their surroundings and being part of their communities.

\subsection{Contribution}
To tackle these issues, our solution would aim at integrating a system into the entire household by using Philips Hue light fixtures or, as a cheaper alternative, providing LED strips, removing the need for a portable device that may be easily forgotten. Furthermore, the system will give the user options on a configuration that works best for their home. On top of that we will provide a low power vibration bracelet solution. This would be sleek so that it can be worn while sleeping while providing a gentle but noticeable vibration. The bracelet could be charged while away from the house, so a user will always be able to use it once in the house. Together, our system will be low cost while providing more functionality towards the hearing impaired.
